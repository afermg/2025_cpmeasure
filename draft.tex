% Created 2025-05-20 Tue 13:14
% Intended LaTeX compiler: pdflatex
\documentclass{article}


%%%%%%%% ICML 2025 EXAMPLE LATEX SUBMISSION FILE %%%%%%%%%%%%%%%%%
\usepackage[T1]{fontenc}
% Recommended, but optional, packages for figures and better typesetting:
\usepackage{microtype}
\usepackage{graphicx}
\usepackage{subfigure}
\usepackage{booktabs} % for professional tables

% hyperref makes hyperlinks in the resulting PDF.
% If your build breaks (sometimes temporarily if a hyperlink spans a page)
% please comment out the following usepackage line and replace
%\usepackage{icml2025} with \usepackage[nohyperref]{icml2025} above.
\usepackage{hyperref}


% Attempt to make hyperref and algorithmic work together better:
\newcommand{\theHalgorithm}{\arabic{algorithm}}

% Use the following line for the initial blind version submitted for review:
%\usepackage{style/icml2025}

% If accepted, instead use the following line for the camera-ready submission:
\usepackage[accepted]{style/icml2025}

% For theorems and such
\usepackage{amsmath}
\usepackage{amssymb}
\usepackage{mathtools}
\usepackage{amsthm}


% if you use cleveref..
\usepackage[capitalize,noabbrev]{cleveref}
%%%%%%%%%%%%%%%%%%%%%%%%%%%%%%%%
% THEOREMS
%%%%%%%%%%%%%%%%%%%%%%%%%%%%%%%%
\theoremstyle{plain}
\newtheorem{theorem}{Theorem}[section]
\newtheorem{proposition}[theorem]{Proposition}
\newtheorem{lemma}[theorem]{Lemma}
\newtheorem{corollary}[theorem]{Corollary}
\theoremstyle{definition}
\newtheorem{definition}[theorem]{Definition}
\newtheorem{assumption}[theorem]{Assumption}
\theoremstyle{remark}
\newtheorem{remark}[theorem]{Remark}

% Todonotes is useful during development; simply uncomment the next line
%    and comment out the line below the next line to turn off comments
%\usepackage[disable,textsize=tiny]{todonotes}
\usepackage[textsize=tiny]{todonotes}

% The \icmltitle you define below is probably too long as a header.
% Therefore, a short form for the running title is supplied here:
\icmltitlerunning{Test short title ICML 2025}


\usepackage[inkscapelatex=false]{svg}
\date{}
\title{}
\hypersetup{
 pdfauthor={Alan Munoz},
 pdftitle={},
 pdfkeywords={},
 pdfsubject={},
 pdfcreator={Emacs 30.1 (Org mode 9.7.11)}, 
 pdflang={English}}
\usepackage{natbib}
\begin{document}

\twocolumn[
\icmltitle{cp\_measure: morphological profiling for data scientists}

% It is OKAY to include author information, even for blind
% submissions: the style file will automatically remove it for you
% unless you've provided the [accepted] option to the icml2025
% package.

% List of affiliations: The first argument should be a (short)
% identifier you will use later to specify author affiliations
% Academic affiliations should list Department, University, City, Region, Country
% Industry affiliations should list Company, City, Region, Country

% You can specify symbols, otherwise they are numbered in order.
% Ideally, you should not use this facility. Affiliations will be numbered
% in order of appearance and this is the preferred way.
\icmlsetsymbol{equal}{*}

\begin{icmlauthorlist}
\icmlauthor{Al\'an F. Mu\~{n}oz}{broad}
\icmlauthor{Tim Treis}{hh,broad}
\icmlauthor{Alexandr A. Kalinin}{broad}
\icmlauthor{Shatavisha Dasgupta}{broad}
\icmlauthor{Fabian Theis}{hh}
\icmlauthor{Anne E. Carpenter}{broad}
\icmlauthor{Shantanu Singh}{broad}
\end{icmlauthorlist}

\icmlaffiliation{broad}{Broad Institute of MIT and Harvard, United States}
\icmlaffiliation{hh}{Institute of Computational Biology, Helmholtz Zentrum München, Germany}

\icmlcorrespondingauthor{Al\'an F. Mu\~{n}oz}{amunozgo@broadinstitute.org}
\icmlcorrespondingauthor{Shantanu Singh}{shantanu@broadinstitute.org}

% You may provide any keywords that you
% find helpful for describing your paper; these are used to populate
% the "keywords" metadata in the PDF but will not be shown in the document
\icmlkeywords{Machine Learning, ICML}

\vskip 0.3in
]

% this must go after the closing bracket ] following \twocolumn[ ...

% This command actually creates the footnote in the first column
% listing the affiliations and the copyright notice.
% The command takes one argument, which is text to display at the start of the footnote.
% The \icmlEqualContribution command is standard text for equal contribution.
% Remove it (just {}) if you do not need this facility.

\printAffiliationsAndNotice{}  % leave blank if no need to mention equal contribution
% \printAffiliationsAndNotice{\icmlEqualContribution} % otherwise use the standard text.

\begin{abstract}
Biological image analysis focuses on measuring a particular visual property of interest for cells or other entities. It 
Quantifying object properties in images is a core challenge in biological imaging. The current tools require significant human intervention. Here we introduce an efficient computational library, cp\_measure, which provides programmatic access to the most commonly used metrics to convert images and objects into features. The features are consistent with those produced by the popular graphical user interface-based software CellProfiler, and we showcase tasks for which cp\_measure is more suitable than alternatives. cp\_measure opens the door to community-driven development and improvement of bioimage analysis metrics and pipelines, increasing the scaling capabilities, reproducibility, and accessibility for computational and data scientists.
\end{abstract}
\section{Introduction}
\label{sec:org4c9ba67}
Modern biologists use a wide array of fluorescence dyes or proteins to observe the location and distribution of cells, organelles, and other components by microscopy.

Historically, the human eyes and brain drew conclusions from biological images but this approach is limited in scale, objectivity, and reproducibility. Now, quantification is standard, with software often identifying regions of interest (such as cells) and calculating metrics that represent these regions, such as intensity distributions.

Morphological profiling is a technique measuring an array of shape and intensity features for all biological objects, such as cells. These features are fed into statistical or machine learning methods to identify biologically meaningful patterns. One of its biggest applications is drug discovery, where scientists leverage microscopy's low acquisition cost and high throughput to accomplish many goals, such as grouping genes by function, identifying chemical compounds that target a protein, and predicting toxicity of drug candidates \citep{sealDecadeSystematicReview2024}. 
\subsection{The current state of bioimage analysis}
\label{sec:org8f5b33d}
The most widely used software for processing high-throughput biological images is CellProfiler \citep{stirlingCellProfiler4Improvements2021}. It can be used by experimental biologists with scant programming experience, for whom it has been a boon. By contrast, for computer-savvy scientists, its graphical user interface and lack of programmatic access means friction in pipelines comprised of multiple tools. CellProfiler is ideal for creating and iteratively adapting manually-defined workflows; for low-customization high-throughput analyses such as extracting features for image-based profiles, however, certain aspects become a hindrance.

From the data-science perspective, there are multiple challenges lessening the attractiveness of CellProfiler. First, the integration of CellProfiler with other tools can be a struggle, as building plug-ins to add functionality is a time and effort-consuming challenge that requires an understanding of its interfaces. Additionally, its usage precludes the development of custom-made image-processing steps, such as applying an image filter before feature extraction. Lastly, CellProfiler depends on many Python and Java packages, increasing the likelihood of dependency conflicts. Containers mitigate this problem, but not without increased complexity and caveats.

Alternative tools can extract features for image-based profiles, such as scikit-image \citep{waltScikitimageImageProcessing2014}, ScaleFex, or SpaCR \citep{comoletHighlyEfficientScalable2024,einarolafssonSpaCr2025} . However, these have limitations: First, their implementation is completely independent from CellProfiler and thus cannot reproduce existing datasets produced using CellProfiler, such as the Cell Painting Gallery, potentially impacting the biological interpretations. Moreover, a common problem is that these tools enforce a structure for the input data and require wrangling with the data before processing a single image. Some alternatives, such as ScaleFex, are designed for cloud usage, neglecting a significant number of cases where local compute suffices. 

\subsection{Engineered features in the era of deep learning}
\label{sec:org9dc3dfa}
An alternative to extracting a defined set of engineered features is to instead use a trained deep learning model to extract features. Results are mixed, with suitably-trained deep learning networks usually outperforming engineered features \cite{lafargeCapturingSingleCellPhenotypic2019,moshkovLearningRepresentationsImagebased2022,chowPredictingDrugPolypharmacology2022,wolfSCANPYLargescaleSinglecell2018}, but not always \cite{tangMorphologicalProfilingDrug2024,kimSelfsupervisionAdvancesMorphological2023}. Training a deep learning model to be a good feature extractor takes significant hands-on and compute time, and often requires appropriate training images. Trained models can in some cases be used on new datasets, eliminating the training step. Compared to engineered features, deep learning approaches have a higher barrier of entry, as most models require annotating data. Foundational models, whose goal is to lower this barrier, fail to generalize well for out-of-distribution data \cite{aza}. Generally deep learning-based feature extraction takes less compute time than CellProfiler does; however, the features are not obviously interpretable and require specialized techniques for biologists to attempt to understand patterns of changes in image-based profiles, which is not always successful \citep{moenDeepLearningCellular2019}.  Unlike learned features, whose interpretation requires complex methods like counterfactual methods or attention maps, most engineered features have a mathematical definition that is faster to calculate and easier to translate into biological concepts. 

In summary, while deep learning-based features have some advantages in some contexts, and while tools to extract engineered features exist, there remains a need for a modular tool that can extract engineered features while readily integrating with computational workflows to achieve reproducible data analyses within a few lines of code, whether locally or in the cloud. We thus believe that there is the need for a readily-integrated feature extraction tool aimed at generalist and high throughput problems not being solved by the current software ecosystem. With all this in mind we developed cp\_measure.


\section{Extracting interpretable features}
\label{sec:org61842b5}
The library cp\_measure is  the CellProfiler codebase, adapted to include calculations for all features of an image-based profiling pipeline, while removing the user interface and anything else non-essential to the task. Measurements are categorized based on three kinds of input: One object + one imaging channel (these can be used to measure such metrics as intensity, texture), one object + multiple channels (e.g., for Pearson correlations between intensity values across channels) and multiple objects (e.g., number of neighbours).

\begin{figure}[htbp]
\centering
\includesvg[width=.9\linewidth]{./figs/cpmeasure_overview}
\caption{\label{fig:overview}cp\_measure generates features from images by using information in every region of interest ("object"). It is commonly used to featurize the pairwise combination of all the available channels and objects. The resultant matrices represent the entire experiment and can be studied using statistical, machine, and deep learning methods.}
\end{figure}

By isolating the feature calculations from the graphical interface and orchestration components, we improved reusability, testability and extensibility. We aim to make the morphological profiling numerical methods accessible to the data science community. Given that the numerical internals are more transparent, we reduced the amount of time and human effort required to integrate it into pipelines, while still retaining the standard features already extracted for numerous public datasets.

%First, we demonstrated the validity and capabilities of cp\_measure in cases that would normally require image analysis knowledge and a considerable amount of time. First, we validate cp\_measure features versus CellProfiler results with a subset of the JUMP dataset \citep{chandrasekaranJUMPCellPainting2023}. Then we showcase cases in which cp\_measure is a more practical choice to process microscopy data: First 3D images of astrocytes and then spatial transcriptomics. These use-cases demonstrate its widespread suitability for different types of problems. 
\subsection{Reproducing CellProfiler measurements}
\label{sec:org09b0cd2}

\begin{figure}[htbp]
\centering
\includesvg[width=.9\linewidth]{./figs/jump_r2_examples}
\caption{\label{fig:cp_vs_cpmeasure}cp\_measure features match their CellProfiler analogs. \textbf{Left panel.} Representative examples comparing CellProfiler feature values to cp\_measure's, generated using matching pairs of masks and images. \textbf{Right panel.} \(R^2\) value of a linear fit for each individual feature, comparing cp\_measure to CellProfiler.}
\end{figure}

We first tested whether cp\_measure features match the original CellProfiler features, for a representative set of 300 images. These correspond to 150 perturbations from the JUMP Cell Painting dataset\citep{chandrasekaranJUMPCellPainting2023}, selecting a representative subset of the most significant phenotypes for a given measurement each. We segmented images to obtain the cells and nuclei using CellProfiler, providing object masks (regions of interest) to CellProfiler and to cp\_measure for feature extraction. Next, we applied cp\_measure on these masks with the original images and mapped the features from cp\_measure to CellProfiler, we calculated a linear fit for each matched feature, plus the \(R^2\) value, indicating how well the correlation fits a linear slope.

The validation of cp\_measure features is shown on Figure \ref{fig:cp_vs_cpmeasure}. The left panel shows examples of the comparison of our features against CellProfiler's. The straight lines demonstrate the recapitulation of measurements from our implementation. A few data points fall outside the diagonals, which hint that some edge-cases are treated differently by either tool. The panel on the right shows the \(R^2\) value of a linear interpolation. Given that this value is directly correlated to the correctness of the implementation, we can see that most of our measurements have a linear relationship, regardless of whether the masks were for nuclei or cytoplasm. This result provides evidence that cp\_measure can be confidently in cases where CellProfiler would be used, and the insights are unlikely to change.
\subsection{Results}
\label{sec:orge5b5c6b}
\subsubsection{Astrocyte 3D data}
\label{sec:org447090b}

To demonstrate its ease of use and the value of interpretable features, we next tested cp\_measure in a classification workflow. We processed 433 three-dimensional (3D) images of astrocytes containing 831 cells \citep{kalinin3DCellNuclear2018}. We preprocessed the profiles following standard procedures \citep{caicedoDataanalysisStrategiesImagebased2017}. Then, we trained a Gradient Boosting classifier to identify the day in which the image of any given cell was acquired, which corresponds to the extent of maturation of the cells. With this we identified which features distinguish cells on the later samples and distinguish subpopulations. Finally, we calculated the Shapley values to get a better understanding of the effects of the drugs on the cells \citep{sundararajanManyShapleyValues2020}.

\begin{figure}[htbp]
\centering
\includesvg[width=.9\linewidth]{./figs/example_shap}
\caption{\label{fig:astrocytes}\textbf{Top panel.} Example pair of astrocyte image and masks. The 3D images were projected over the z-axis, taking the maximum value across the z-stack. \textbf{Bottom panel.} Shapley values of the most important features to classify the day in which an image was taken (out of three). The test data accuracy is shown in bold.}
\end{figure}

Figure \ref{fig:astrocytes} shows an example image and its corresponding object masks alongside the Shapley values of a classifier trained on our features. This indicates that the major axis length of the cell to is an indicator of phenotypic effect over the course of the experiment, implying that cells became more elongated on their minor axis. Though unsurprising, as we expect astrocytes to extend over time, cp\_measure was able to uncover this with only a few lines of code.
\subsubsection{Spatial transcriptomics}
\label{sec:org5711d86}
A key advantage of providing these measurements as a standalone Python package is their ease of integration into diverse analytical workflows, which otherwise would require substantial adaptation to the standard CellProfiler environment. The recent proliferation of black-box foundation models trained solely on morphological data highlights morphology as a highly informative and predictive modality. However, the feature vectors produced by these models are typically not interpretable, preventing direct biological assessment. In contrast, classical morphological measurements yield explicit, interpretable readouts -- for instance, the co-localization of fluorescent markers -- facilitating clear biological interpretations.

To demonstrate this utility, we integrated our cp\_measure-based feature extraction into the widely used spatial analysis library Squidpy \citep{pallaSquidpyScalableFramework2022}. Being standalone allowed seamless incorporation into workflows powered by the robust SpatialData \citep{marconatoSpatialDataOpenUniversal2025} framework underlying Squidpy. Because spatial datasets often comprise significantly more cells per field-of-view (FOV) than conventional microscopy screenings -- up to approximately 100,000 cells-traditional software typically cannot process these large images without cropping, which introduces boundary artefacts. Leveraging the modular design of cp\_measure, we parallelized feature extraction at the single-cell level, streaming batches of cells across computational cores. This approach enables efficient computation even on large-scale datasets, a feat not achievable with standard CellProfiler software.

To further illustrate the value of morphological features, we evaluated their impact on cell-type prediction tasks using spatial transcriptomics data. This application is particularly compelling, as current spatial transcriptomics technologies typically produce matched histological images that remain largely underutilized beyond visualization. We analysed two mouse brain datasets generated by Bruker Spatial's CosMx platform \citep{CosMxSMIMouse2025}. Each dataset comprises expression profiles for 960 genes and immunofluorescence images captured via five distinct fluorescent probes ('Histone', 'DNA', 'GFAP', 'G', 'rRNA'). Morphological features were extracted from these 5-channel images for both datasets. Subsequently, both gene expression and morphological data were preprocessed according to best practices established by Scanpy \citep{wolfSCANPYLargescaleSinglecell2018} and Pycytominer \citep{serranoReproducibleImagebasedProfiling2025} respectively. We trained an XGBoost model to predict cell types on the larger dataset (48,556 cells; see Fig. XXX, panel XXX), comparing models using either gene expression alone or combined gene expression and morphological data. Model performance was assessed by predicting cell types in a smaller independent dataset (38,996 cells), using the F1-score metric stratified by cell type. Figure XXX (panel XXX) highlights the improved predictive accuracy obtained when morphological features are included. Importantly, this performance enhancement required no additional experimental effort, underscoring the benefit of employing cp\_measure beyond its traditional scope.

\begin{figure}[htbp]
\centering
\includegraphics[width=.9\linewidth]{./figs/spatial.png}
\caption{\label{fig:spatial_omics}{[}PLACEHOLDER] Spatial omics analysis.}
\end{figure}
\section{Discussion}
\label{sec:orgf37b369}
While point-and-click interfaces open the world of image analysis to many researchers, they can present barriers for smooth adoption into computational workflows. In this work we introduced our new library cp\_measure, which calculates a set of widely used engineered features relevant to whole images and to regions of interest (object masks). It enables simpler automated analyses of microscopy data in either short scripts and complex pipelines. It avoids graphical interfaces to process microscopy data, resulting in better scaling capabilities for high-content microscopy, with or without cloud infrastructure.

The biologically interpretable features provided by cp\_measure complement deep learning features and offer a better mechanistic understanding of the underlying biology. When used in tandem with generalist image-processing tools, such as Cellpose for segmentation (CITE Cellpose), we can streamline machine and deep learning workflows. 

Already, cp\_measure  has been used in multiple experiments, including in non-biological contexts such as environmental monitoring \citep{ideharaExploringNileRed2025}. We therefore foresee cp\_measure being useful for a broad scientific community beyond morphological profiling.
\section{Future work}
\label{sec:org5cdbb12}
The most obvious way to make cp\_measure more useful is to contribute it back to CellProfiler. This would ensure that the results from pipelines built with either tool will always be comparable, while also providing the opportunity to formalize the inputs and outputs of all measurements. 

We also plan to develop a comprehensive test suite to guarantee mathematical correctness, which currently CellProfiler itself is lacking. A functioning test suite would enable more confident and expedient optimization for the most compute-consuming features, such as granularity. Once tests are in place, we could add to support just-in-time compiling and GPUs. Finally, we envision the cp\_measure library could be the place to develop and distribute new measurements from the community. 

\bibliographystyle{icml2025}
\bibliography{bibliography}
\section{Appendix}
\label{sec:orgdd18dd8}
\subsection{Methods}
\label{sec:orgb3e9382}
\subsubsection{Data and software}
\label{sec:orgbda0ae2}
The code for cp\_measure is available on \url{https://anonymous.4open.science/r/cp\_measure-B0DA}. All code to reproduce the analyses and figures, alongside links to the original data, is available on the GitHub repository \url{https://github.com/afermg/2025\_cpmeasure/}. The datasets we produced for this work are available on Zenodo, and the latest version can be found on \url{https://zenodo.org/records/15390631/latest}.
\subsubsection{Data and software}
\label{sec:acknowledgements}

This work was supported by GSK and the National Institutes of Health, grant (R35 GM122547 to AEC). The authors would like to thank Nodar Gogoberidze for his valuable insight into CellProfiler's code base, and Minh Doan and Eliot McKinley for their valuable feedback and discussions during the course of this project.
\end{document}
